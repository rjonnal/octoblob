
% Default to the notebook output style

    


% Inherit from the specified cell style.




    
\documentclass[11pt]{article}

    
    
    \usepackage[T1]{fontenc}
    % Nicer default font (+ math font) than Computer Modern for most use cases
    \usepackage{mathpazo}

    % Basic figure setup, for now with no caption control since it's done
    % automatically by Pandoc (which extracts ![](path) syntax from Markdown).
    \usepackage{graphicx}
    % We will generate all images so they have a width \maxwidth. This means
    % that they will get their normal width if they fit onto the page, but
    % are scaled down if they would overflow the margins.
    \makeatletter
    \def\maxwidth{\ifdim\Gin@nat@width>\linewidth\linewidth
    \else\Gin@nat@width\fi}
    \makeatother
    \let\Oldincludegraphics\includegraphics
    % Set max figure width to be 80% of text width, for now hardcoded.
    \renewcommand{\includegraphics}[1]{\Oldincludegraphics[width=.8\maxwidth]{#1}}
    % Ensure that by default, figures have no caption (until we provide a
    % proper Figure object with a Caption API and a way to capture that
    % in the conversion process - todo).
    \usepackage{caption}
    \DeclareCaptionLabelFormat{nolabel}{}
    \captionsetup{labelformat=nolabel}

    \usepackage{adjustbox} % Used to constrain images to a maximum size 
    \usepackage{xcolor} % Allow colors to be defined
    \usepackage{enumerate} % Needed for markdown enumerations to work
    \usepackage{geometry} % Used to adjust the document margins
    \usepackage{amsmath} % Equations
    \usepackage{amssymb} % Equations
    \usepackage{textcomp} % defines textquotesingle
    % Hack from http://tex.stackexchange.com/a/47451/13684:
    \AtBeginDocument{%
        \def\PYZsq{\textquotesingle}% Upright quotes in Pygmentized code
    }
    \usepackage{upquote} % Upright quotes for verbatim code
    \usepackage{eurosym} % defines \euro
    \usepackage[mathletters]{ucs} % Extended unicode (utf-8) support
    \usepackage[utf8x]{inputenc} % Allow utf-8 characters in the tex document
    \usepackage{fancyvrb} % verbatim replacement that allows latex
    \usepackage{grffile} % extends the file name processing of package graphics 
                         % to support a larger range 
    % The hyperref package gives us a pdf with properly built
    % internal navigation ('pdf bookmarks' for the table of contents,
    % internal cross-reference links, web links for URLs, etc.)
    \usepackage{hyperref}
    \usepackage{longtable} % longtable support required by pandoc >1.10
    \usepackage{booktabs}  % table support for pandoc > 1.12.2
    \usepackage[inline]{enumitem} % IRkernel/repr support (it uses the enumerate* environment)
    \usepackage[normalem]{ulem} % ulem is needed to support strikethroughs (\sout)
                                % normalem makes italics be italics, not underlines
    

    
    
    % Colors for the hyperref package
    \definecolor{urlcolor}{rgb}{0,.145,.698}
    \definecolor{linkcolor}{rgb}{.71,0.21,0.01}
    \definecolor{citecolor}{rgb}{.12,.54,.11}

    % ANSI colors
    \definecolor{ansi-black}{HTML}{3E424D}
    \definecolor{ansi-black-intense}{HTML}{282C36}
    \definecolor{ansi-red}{HTML}{E75C58}
    \definecolor{ansi-red-intense}{HTML}{B22B31}
    \definecolor{ansi-green}{HTML}{00A250}
    \definecolor{ansi-green-intense}{HTML}{007427}
    \definecolor{ansi-yellow}{HTML}{DDB62B}
    \definecolor{ansi-yellow-intense}{HTML}{B27D12}
    \definecolor{ansi-blue}{HTML}{208FFB}
    \definecolor{ansi-blue-intense}{HTML}{0065CA}
    \definecolor{ansi-magenta}{HTML}{D160C4}
    \definecolor{ansi-magenta-intense}{HTML}{A03196}
    \definecolor{ansi-cyan}{HTML}{60C6C8}
    \definecolor{ansi-cyan-intense}{HTML}{258F8F}
    \definecolor{ansi-white}{HTML}{C5C1B4}
    \definecolor{ansi-white-intense}{HTML}{A1A6B2}

    % commands and environments needed by pandoc snippets
    % extracted from the output of `pandoc -s`
    \providecommand{\tightlist}{%
      \setlength{\itemsep}{0pt}\setlength{\parskip}{0pt}}
    \DefineVerbatimEnvironment{Highlighting}{Verbatim}{commandchars=\\\{\}}
    % Add ',fontsize=\small' for more characters per line
    \newenvironment{Shaded}{}{}
    \newcommand{\KeywordTok}[1]{\textcolor[rgb]{0.00,0.44,0.13}{\textbf{{#1}}}}
    \newcommand{\DataTypeTok}[1]{\textcolor[rgb]{0.56,0.13,0.00}{{#1}}}
    \newcommand{\DecValTok}[1]{\textcolor[rgb]{0.25,0.63,0.44}{{#1}}}
    \newcommand{\BaseNTok}[1]{\textcolor[rgb]{0.25,0.63,0.44}{{#1}}}
    \newcommand{\FloatTok}[1]{\textcolor[rgb]{0.25,0.63,0.44}{{#1}}}
    \newcommand{\CharTok}[1]{\textcolor[rgb]{0.25,0.44,0.63}{{#1}}}
    \newcommand{\StringTok}[1]{\textcolor[rgb]{0.25,0.44,0.63}{{#1}}}
    \newcommand{\CommentTok}[1]{\textcolor[rgb]{0.38,0.63,0.69}{\textit{{#1}}}}
    \newcommand{\OtherTok}[1]{\textcolor[rgb]{0.00,0.44,0.13}{{#1}}}
    \newcommand{\AlertTok}[1]{\textcolor[rgb]{1.00,0.00,0.00}{\textbf{{#1}}}}
    \newcommand{\FunctionTok}[1]{\textcolor[rgb]{0.02,0.16,0.49}{{#1}}}
    \newcommand{\RegionMarkerTok}[1]{{#1}}
    \newcommand{\ErrorTok}[1]{\textcolor[rgb]{1.00,0.00,0.00}{\textbf{{#1}}}}
    \newcommand{\NormalTok}[1]{{#1}}
    
    % Additional commands for more recent versions of Pandoc
    \newcommand{\ConstantTok}[1]{\textcolor[rgb]{0.53,0.00,0.00}{{#1}}}
    \newcommand{\SpecialCharTok}[1]{\textcolor[rgb]{0.25,0.44,0.63}{{#1}}}
    \newcommand{\VerbatimStringTok}[1]{\textcolor[rgb]{0.25,0.44,0.63}{{#1}}}
    \newcommand{\SpecialStringTok}[1]{\textcolor[rgb]{0.73,0.40,0.53}{{#1}}}
    \newcommand{\ImportTok}[1]{{#1}}
    \newcommand{\DocumentationTok}[1]{\textcolor[rgb]{0.73,0.13,0.13}{\textit{{#1}}}}
    \newcommand{\AnnotationTok}[1]{\textcolor[rgb]{0.38,0.63,0.69}{\textbf{\textit{{#1}}}}}
    \newcommand{\CommentVarTok}[1]{\textcolor[rgb]{0.38,0.63,0.69}{\textbf{\textit{{#1}}}}}
    \newcommand{\VariableTok}[1]{\textcolor[rgb]{0.10,0.09,0.49}{{#1}}}
    \newcommand{\ControlFlowTok}[1]{\textcolor[rgb]{0.00,0.44,0.13}{\textbf{{#1}}}}
    \newcommand{\OperatorTok}[1]{\textcolor[rgb]{0.40,0.40,0.40}{{#1}}}
    \newcommand{\BuiltInTok}[1]{{#1}}
    \newcommand{\ExtensionTok}[1]{{#1}}
    \newcommand{\PreprocessorTok}[1]{\textcolor[rgb]{0.74,0.48,0.00}{{#1}}}
    \newcommand{\AttributeTok}[1]{\textcolor[rgb]{0.49,0.56,0.16}{{#1}}}
    \newcommand{\InformationTok}[1]{\textcolor[rgb]{0.38,0.63,0.69}{\textbf{\textit{{#1}}}}}
    \newcommand{\WarningTok}[1]{\textcolor[rgb]{0.38,0.63,0.69}{\textbf{\textit{{#1}}}}}
    
    
    % Define a nice break command that doesn't care if a line doesn't already
    % exist.
    \def\br{\hspace*{\fill} \\* }
    % Math Jax compatability definitions
    \def\gt{>}
    \def\lt{<}
    % Document parameters
    \title{octoblob\_OCTA\_pipeline\_example}
    
    
    

    % Pygments definitions
    
\makeatletter
\def\PY@reset{\let\PY@it=\relax \let\PY@bf=\relax%
    \let\PY@ul=\relax \let\PY@tc=\relax%
    \let\PY@bc=\relax \let\PY@ff=\relax}
\def\PY@tok#1{\csname PY@tok@#1\endcsname}
\def\PY@toks#1+{\ifx\relax#1\empty\else%
    \PY@tok{#1}\expandafter\PY@toks\fi}
\def\PY@do#1{\PY@bc{\PY@tc{\PY@ul{%
    \PY@it{\PY@bf{\PY@ff{#1}}}}}}}
\def\PY#1#2{\PY@reset\PY@toks#1+\relax+\PY@do{#2}}

\expandafter\def\csname PY@tok@gd\endcsname{\def\PY@tc##1{\textcolor[rgb]{0.63,0.00,0.00}{##1}}}
\expandafter\def\csname PY@tok@gu\endcsname{\let\PY@bf=\textbf\def\PY@tc##1{\textcolor[rgb]{0.50,0.00,0.50}{##1}}}
\expandafter\def\csname PY@tok@gt\endcsname{\def\PY@tc##1{\textcolor[rgb]{0.00,0.27,0.87}{##1}}}
\expandafter\def\csname PY@tok@gs\endcsname{\let\PY@bf=\textbf}
\expandafter\def\csname PY@tok@gr\endcsname{\def\PY@tc##1{\textcolor[rgb]{1.00,0.00,0.00}{##1}}}
\expandafter\def\csname PY@tok@cm\endcsname{\let\PY@it=\textit\def\PY@tc##1{\textcolor[rgb]{0.25,0.50,0.50}{##1}}}
\expandafter\def\csname PY@tok@vg\endcsname{\def\PY@tc##1{\textcolor[rgb]{0.10,0.09,0.49}{##1}}}
\expandafter\def\csname PY@tok@vi\endcsname{\def\PY@tc##1{\textcolor[rgb]{0.10,0.09,0.49}{##1}}}
\expandafter\def\csname PY@tok@vm\endcsname{\def\PY@tc##1{\textcolor[rgb]{0.10,0.09,0.49}{##1}}}
\expandafter\def\csname PY@tok@mh\endcsname{\def\PY@tc##1{\textcolor[rgb]{0.40,0.40,0.40}{##1}}}
\expandafter\def\csname PY@tok@cs\endcsname{\let\PY@it=\textit\def\PY@tc##1{\textcolor[rgb]{0.25,0.50,0.50}{##1}}}
\expandafter\def\csname PY@tok@ge\endcsname{\let\PY@it=\textit}
\expandafter\def\csname PY@tok@vc\endcsname{\def\PY@tc##1{\textcolor[rgb]{0.10,0.09,0.49}{##1}}}
\expandafter\def\csname PY@tok@il\endcsname{\def\PY@tc##1{\textcolor[rgb]{0.40,0.40,0.40}{##1}}}
\expandafter\def\csname PY@tok@go\endcsname{\def\PY@tc##1{\textcolor[rgb]{0.53,0.53,0.53}{##1}}}
\expandafter\def\csname PY@tok@cp\endcsname{\def\PY@tc##1{\textcolor[rgb]{0.74,0.48,0.00}{##1}}}
\expandafter\def\csname PY@tok@gi\endcsname{\def\PY@tc##1{\textcolor[rgb]{0.00,0.63,0.00}{##1}}}
\expandafter\def\csname PY@tok@gh\endcsname{\let\PY@bf=\textbf\def\PY@tc##1{\textcolor[rgb]{0.00,0.00,0.50}{##1}}}
\expandafter\def\csname PY@tok@ni\endcsname{\let\PY@bf=\textbf\def\PY@tc##1{\textcolor[rgb]{0.60,0.60,0.60}{##1}}}
\expandafter\def\csname PY@tok@nl\endcsname{\def\PY@tc##1{\textcolor[rgb]{0.63,0.63,0.00}{##1}}}
\expandafter\def\csname PY@tok@nn\endcsname{\let\PY@bf=\textbf\def\PY@tc##1{\textcolor[rgb]{0.00,0.00,1.00}{##1}}}
\expandafter\def\csname PY@tok@no\endcsname{\def\PY@tc##1{\textcolor[rgb]{0.53,0.00,0.00}{##1}}}
\expandafter\def\csname PY@tok@na\endcsname{\def\PY@tc##1{\textcolor[rgb]{0.49,0.56,0.16}{##1}}}
\expandafter\def\csname PY@tok@nb\endcsname{\def\PY@tc##1{\textcolor[rgb]{0.00,0.50,0.00}{##1}}}
\expandafter\def\csname PY@tok@nc\endcsname{\let\PY@bf=\textbf\def\PY@tc##1{\textcolor[rgb]{0.00,0.00,1.00}{##1}}}
\expandafter\def\csname PY@tok@nd\endcsname{\def\PY@tc##1{\textcolor[rgb]{0.67,0.13,1.00}{##1}}}
\expandafter\def\csname PY@tok@ne\endcsname{\let\PY@bf=\textbf\def\PY@tc##1{\textcolor[rgb]{0.82,0.25,0.23}{##1}}}
\expandafter\def\csname PY@tok@nf\endcsname{\def\PY@tc##1{\textcolor[rgb]{0.00,0.00,1.00}{##1}}}
\expandafter\def\csname PY@tok@si\endcsname{\let\PY@bf=\textbf\def\PY@tc##1{\textcolor[rgb]{0.73,0.40,0.53}{##1}}}
\expandafter\def\csname PY@tok@s2\endcsname{\def\PY@tc##1{\textcolor[rgb]{0.73,0.13,0.13}{##1}}}
\expandafter\def\csname PY@tok@nt\endcsname{\let\PY@bf=\textbf\def\PY@tc##1{\textcolor[rgb]{0.00,0.50,0.00}{##1}}}
\expandafter\def\csname PY@tok@nv\endcsname{\def\PY@tc##1{\textcolor[rgb]{0.10,0.09,0.49}{##1}}}
\expandafter\def\csname PY@tok@s1\endcsname{\def\PY@tc##1{\textcolor[rgb]{0.73,0.13,0.13}{##1}}}
\expandafter\def\csname PY@tok@dl\endcsname{\def\PY@tc##1{\textcolor[rgb]{0.73,0.13,0.13}{##1}}}
\expandafter\def\csname PY@tok@ch\endcsname{\let\PY@it=\textit\def\PY@tc##1{\textcolor[rgb]{0.25,0.50,0.50}{##1}}}
\expandafter\def\csname PY@tok@m\endcsname{\def\PY@tc##1{\textcolor[rgb]{0.40,0.40,0.40}{##1}}}
\expandafter\def\csname PY@tok@gp\endcsname{\let\PY@bf=\textbf\def\PY@tc##1{\textcolor[rgb]{0.00,0.00,0.50}{##1}}}
\expandafter\def\csname PY@tok@sh\endcsname{\def\PY@tc##1{\textcolor[rgb]{0.73,0.13,0.13}{##1}}}
\expandafter\def\csname PY@tok@ow\endcsname{\let\PY@bf=\textbf\def\PY@tc##1{\textcolor[rgb]{0.67,0.13,1.00}{##1}}}
\expandafter\def\csname PY@tok@sx\endcsname{\def\PY@tc##1{\textcolor[rgb]{0.00,0.50,0.00}{##1}}}
\expandafter\def\csname PY@tok@bp\endcsname{\def\PY@tc##1{\textcolor[rgb]{0.00,0.50,0.00}{##1}}}
\expandafter\def\csname PY@tok@c1\endcsname{\let\PY@it=\textit\def\PY@tc##1{\textcolor[rgb]{0.25,0.50,0.50}{##1}}}
\expandafter\def\csname PY@tok@fm\endcsname{\def\PY@tc##1{\textcolor[rgb]{0.00,0.00,1.00}{##1}}}
\expandafter\def\csname PY@tok@o\endcsname{\def\PY@tc##1{\textcolor[rgb]{0.40,0.40,0.40}{##1}}}
\expandafter\def\csname PY@tok@kc\endcsname{\let\PY@bf=\textbf\def\PY@tc##1{\textcolor[rgb]{0.00,0.50,0.00}{##1}}}
\expandafter\def\csname PY@tok@c\endcsname{\let\PY@it=\textit\def\PY@tc##1{\textcolor[rgb]{0.25,0.50,0.50}{##1}}}
\expandafter\def\csname PY@tok@mf\endcsname{\def\PY@tc##1{\textcolor[rgb]{0.40,0.40,0.40}{##1}}}
\expandafter\def\csname PY@tok@err\endcsname{\def\PY@bc##1{\setlength{\fboxsep}{0pt}\fcolorbox[rgb]{1.00,0.00,0.00}{1,1,1}{\strut ##1}}}
\expandafter\def\csname PY@tok@mb\endcsname{\def\PY@tc##1{\textcolor[rgb]{0.40,0.40,0.40}{##1}}}
\expandafter\def\csname PY@tok@ss\endcsname{\def\PY@tc##1{\textcolor[rgb]{0.10,0.09,0.49}{##1}}}
\expandafter\def\csname PY@tok@sr\endcsname{\def\PY@tc##1{\textcolor[rgb]{0.73,0.40,0.53}{##1}}}
\expandafter\def\csname PY@tok@mo\endcsname{\def\PY@tc##1{\textcolor[rgb]{0.40,0.40,0.40}{##1}}}
\expandafter\def\csname PY@tok@kd\endcsname{\let\PY@bf=\textbf\def\PY@tc##1{\textcolor[rgb]{0.00,0.50,0.00}{##1}}}
\expandafter\def\csname PY@tok@mi\endcsname{\def\PY@tc##1{\textcolor[rgb]{0.40,0.40,0.40}{##1}}}
\expandafter\def\csname PY@tok@kn\endcsname{\let\PY@bf=\textbf\def\PY@tc##1{\textcolor[rgb]{0.00,0.50,0.00}{##1}}}
\expandafter\def\csname PY@tok@cpf\endcsname{\let\PY@it=\textit\def\PY@tc##1{\textcolor[rgb]{0.25,0.50,0.50}{##1}}}
\expandafter\def\csname PY@tok@kr\endcsname{\let\PY@bf=\textbf\def\PY@tc##1{\textcolor[rgb]{0.00,0.50,0.00}{##1}}}
\expandafter\def\csname PY@tok@s\endcsname{\def\PY@tc##1{\textcolor[rgb]{0.73,0.13,0.13}{##1}}}
\expandafter\def\csname PY@tok@kp\endcsname{\def\PY@tc##1{\textcolor[rgb]{0.00,0.50,0.00}{##1}}}
\expandafter\def\csname PY@tok@w\endcsname{\def\PY@tc##1{\textcolor[rgb]{0.73,0.73,0.73}{##1}}}
\expandafter\def\csname PY@tok@kt\endcsname{\def\PY@tc##1{\textcolor[rgb]{0.69,0.00,0.25}{##1}}}
\expandafter\def\csname PY@tok@sc\endcsname{\def\PY@tc##1{\textcolor[rgb]{0.73,0.13,0.13}{##1}}}
\expandafter\def\csname PY@tok@sb\endcsname{\def\PY@tc##1{\textcolor[rgb]{0.73,0.13,0.13}{##1}}}
\expandafter\def\csname PY@tok@sa\endcsname{\def\PY@tc##1{\textcolor[rgb]{0.73,0.13,0.13}{##1}}}
\expandafter\def\csname PY@tok@k\endcsname{\let\PY@bf=\textbf\def\PY@tc##1{\textcolor[rgb]{0.00,0.50,0.00}{##1}}}
\expandafter\def\csname PY@tok@se\endcsname{\let\PY@bf=\textbf\def\PY@tc##1{\textcolor[rgb]{0.73,0.40,0.13}{##1}}}
\expandafter\def\csname PY@tok@sd\endcsname{\let\PY@it=\textit\def\PY@tc##1{\textcolor[rgb]{0.73,0.13,0.13}{##1}}}

\def\PYZbs{\char`\\}
\def\PYZus{\char`\_}
\def\PYZob{\char`\{}
\def\PYZcb{\char`\}}
\def\PYZca{\char`\^}
\def\PYZam{\char`\&}
\def\PYZlt{\char`\<}
\def\PYZgt{\char`\>}
\def\PYZsh{\char`\#}
\def\PYZpc{\char`\%}
\def\PYZdl{\char`\$}
\def\PYZhy{\char`\-}
\def\PYZsq{\char`\'}
\def\PYZdq{\char`\"}
\def\PYZti{\char`\~}
% for compatibility with earlier versions
\def\PYZat{@}
\def\PYZlb{[}
\def\PYZrb{]}
\makeatother


    % Exact colors from NB
    \definecolor{incolor}{rgb}{0.0, 0.0, 0.5}
    \definecolor{outcolor}{rgb}{0.545, 0.0, 0.0}



    
    % Prevent overflowing lines due to hard-to-break entities
    \sloppy 
    % Setup hyperref package
    \hypersetup{
      breaklinks=true,  % so long urls are correctly broken across lines
      colorlinks=true,
      urlcolor=urlcolor,
      linkcolor=linkcolor,
      citecolor=citecolor,
      }
    % Slightly bigger margins than the latex defaults
    
    \geometry{verbose,tmargin=1in,bmargin=1in,lmargin=1in,rmargin=1in}
    
    

    \begin{document}
    
    
    \maketitle
    
    

    
    \hypertarget{introduction-to-octa-processing-code}{%
\section{Introduction to OCTA processing
code}\label{introduction-to-octa-processing-code}}

    This document contains instructions for interactively running OCTA
processing. It is meant to illustrate the architecture of the processing
tool chain, the distinction between the OCT/OCTA libraries and
processing scripts, and other important (and confusing) issues. In
actuality, the OCTA data will be processed using Python scripts (i.e.,
batch processing), with no interaction with the user.

The first step in constructing a script is to import the tools you'll
need. \texttt{numpy} and \texttt{matplotlib} are the standard numerical
and plotting libraries in Python, and are always imported. The
\texttt{\_\_future\_\_\_} and \texttt{builtins} imports implement some
Python 3 functions, which will make porting this to Python 3 easier.

\texttt{octoblob} is the unfortunate name I've chosen for the OCT/OCTA
processing libraries. It is a descendent of the now obsolete
\texttt{octopod} and \texttt{cuttlefish} libraries we've used in the
past. We could have imported all the classes and functions from octoblob
with \texttt{from\ octoblob\ import\ *}, but it's better practice to
keep the module name around, so that when module functions are called
(e.g. \texttt{bscan\ =\ blob.make\_bscan(data)}), it's clear that the
function is coming from the octoblob package, and clear where one needs
to go to find the definition of the function.

    \begin{Verbatim}[commandchars=\\\{\}]
{\color{incolor}In [{\color{incolor}1}]:} \PY{k+kn}{from} \PY{n+nn}{\PYZus{}\PYZus{}future\PYZus{}\PYZus{}} \PY{k+kn}{import} \PY{p}{(}\PY{n}{absolute\PYZus{}import}\PY{p}{,} \PY{n}{division}\PY{p}{,}
                                \PY{n}{print\PYZus{}function}\PY{p}{,} \PY{n}{unicode\PYZus{}literals}\PY{p}{)}
        \PY{k+kn}{from} \PY{n+nn}{builtins} \PY{k+kn}{import} \PY{o}{*}
        \PY{k+kn}{import} \PY{n+nn}{numpy} \PY{k+kn}{as} \PY{n+nn}{np}
        \PY{k+kn}{from} \PY{n+nn}{matplotlib} \PY{k+kn}{import} \PY{n}{pyplot} \PY{k}{as} \PY{n}{plt}
        \PY{k+kn}{import} \PY{n+nn}{octoblob} \PY{k+kn}{as} \PY{n+nn}{blob}
\end{Verbatim}


    \hypertarget{some-architectural-principles}{%
\subsection{Some architectural
principles}\label{some-architectural-principles}}

\begin{enumerate}
\def\labelenumi{\arabic{enumi}.}
\item
  One reasonable way to think about scientific software is to split it
  into two categories: \textbf{libraries} and \textbf{scripts}.
  Libraries are collections of functions (and \emph{classes}--more on
  that later) where each function and class has a well-defined goal, and
  the implementations have been extensively tested or otherwise verified
  to be correct. We \emph{really} don't want any errors in a library.
  Scripts are the day-to-day programs we run. Some are batch scripts
  that process lots of data autonomously, and others are exploratory,
  where we run them to see what the data looks like, often in order to
  help us design the next step in the processing pipeline. Sometimes a
  portion of a script becomes used so frequently that it makes sense to
  turn it into a library functions and thus simplify the script.
\item
  Specifically with regard to the OCT/OCTA processing pipeline, I
  believe the libraries should be split into two parts: 1) a library for
  reading raw data and getting it organized, and 2) a library for
  turning raw data into OCT/OCTA images. The first of these is handled
  by a \emph{class}, and the second is handled by a set of
  \emph{functions}.
\item
  \textbf{Classes}. If you're not familiar with object-oriented
  programming, all you need to know about a class is that it is a
  specification for an \emph{object}, i.e.~a list of functions and
  variables that are stored together and somehow insulated from the rest
  of the code. The raw OCT data is handled by a class, because it needs
  to keep track of lots of information about the data. We create an
  \texttt{OCTRawData} class, and it keeps track of how many bytes there
  are per pixel, how many pixels per spectrum, how many spectra per
  B-scan, etc. By implementing this with a class, we don't have to
  specify how to get the raw data every time we need a new frame. We
  just instantiate the object and then ask it for frames, which will be
  illustrated below.
\end{enumerate}

    \hypertarget{parameters-for-the-octrawdata-class}{%
\subsection{\texorpdfstring{Parameters for the \texttt{OCTRawData}
class}{Parameters for the OCTRawData class}}\label{parameters-for-the-octrawdata-class}}

The \texttt{OCTRawData} class needs to know how to get a frame out of
the file, and to do that it needs a bunch of parameters. Let's specify
these first. They should be self-explanatory, but trailing comments may
clarify in some cases.

    \begin{Verbatim}[commandchars=\\\{\}]
{\color{incolor}In [{\color{incolor}2}]:} \PY{c+c1}{\PYZsh{} PARAMETERS FOR RAW DATA SOURCE}
        \PY{n}{filename} \PY{o}{=} \PY{l+s+s1}{\PYZsq{}}\PY{l+s+s1}{../octa\PYZus{}test\PYZus{}set.unp}\PY{l+s+s1}{\PYZsq{}} \PY{c+c1}{\PYZsh{} name of the raw data file}
        \PY{n}{n\PYZus{}vol} \PY{o}{=} \PY{l+m+mi}{1} \PY{c+c1}{\PYZsh{} number of volumes}
        \PY{n}{n\PYZus{}slow} \PY{o}{=} \PY{l+m+mi}{4} \PY{c+c1}{\PYZsh{} number of B\PYZhy{}scans in each volume}
        \PY{n}{n\PYZus{}repeats} \PY{o}{=} \PY{l+m+mi}{5} \PY{c+c1}{\PYZsh{} number of repeats for OCTA data}
        \PY{n}{n\PYZus{}fast} \PY{o}{=} \PY{l+m+mi}{2500} \PY{c+c1}{\PYZsh{} number of A\PYZhy{}scans per B\PYZhy{}scan x number of repeats}
        \PY{n}{n\PYZus{}skip} \PY{o}{=} \PY{l+m+mi}{500} \PY{c+c1}{\PYZsh{} number of A\PYZhy{}scans to skip at the start}
        \PY{n}{n\PYZus{}depth} \PY{o}{=} \PY{l+m+mi}{1536} \PY{c+c1}{\PYZsh{} number of pixels per spectrum}
        \PY{n}{bit\PYZus{}shift\PYZus{}right} \PY{o}{=} \PY{l+m+mi}{4} \PY{c+c1}{\PYZsh{} ignore for now}
        \PY{n}{dtype}\PY{o}{=}\PY{n}{np}\PY{o}{.}\PY{n}{uint16} \PY{c+c1}{\PYZsh{} the data type of the raw data}
        
        \PY{n}{fbg\PYZus{}position} \PY{o}{=} \PY{l+m+mi}{148} \PY{c+c1}{\PYZsh{} if there is an FBG, approximately where is it located}
        \PY{n}{spectrum\PYZus{}start} \PY{o}{=} \PY{l+m+mi}{159} \PY{c+c1}{\PYZsh{} where does the spectral data start (i.e. after FBG)}
        \PY{n}{spectrum\PYZus{}end} \PY{o}{=} \PY{l+m+mi}{1459} \PY{c+c1}{\PYZsh{} where does the spectral data end (i.e., before any dead/dark time at the end)}
\end{Verbatim}


    Now we can instantiate the \texttt{OCTRawData} object, which will later
be used to get frames.

    \begin{Verbatim}[commandchars=\\\{\}]
{\color{incolor}In [{\color{incolor}3}]:} \PY{n}{src} \PY{o}{=} \PY{n}{blob}\PY{o}{.}\PY{n}{OCTRawData}\PY{p}{(}\PY{n}{filename}\PY{p}{,}\PY{n}{n\PYZus{}vol}\PY{p}{,}\PY{n}{n\PYZus{}slow}\PY{p}{,}\PY{n}{n\PYZus{}fast}\PY{p}{,}\PY{n}{n\PYZus{}depth}\PY{p}{,}\PY{n}{n\PYZus{}repeats}\PY{p}{,}
                              \PY{n}{fbg\PYZus{}position}\PY{o}{=}\PY{n}{fbg\PYZus{}position}\PY{p}{,}
                              \PY{n}{spectrum\PYZus{}start}\PY{o}{=}\PY{n}{spectrum\PYZus{}start}\PY{p}{,}\PY{n}{spectrum\PYZus{}end}\PY{o}{=}\PY{n}{spectrum\PYZus{}end}\PY{p}{,}
                              \PY{n}{bit\PYZus{}shift\PYZus{}right}\PY{o}{=}\PY{n}{bit\PYZus{}shift\PYZus{}right}\PY{p}{,}
                              \PY{n}{n\PYZus{}skip}\PY{o}{=}\PY{n}{n\PYZus{}skip}\PY{p}{,}\PY{n}{dtype}\PY{o}{=}\PY{n}{dtype}\PY{p}{)}
\end{Verbatim}


    \begin{Verbatim}[commandchars=\\\{\}]
File size incorrect.
30720000	expected
32256000	actual
n\_vol		1
n\_slow		4
n\_repeats	5
n\_fast		2500
n\_depth		1536
bytes\_per\_pixel	2
total\_expected\_size	30720000

    \end{Verbatim}

    The ``File size incorrect'' warning is just telling us that there are
more bytes in the file than we need. This is because using Yifan's
software and the Axsun source, there's no synchronization between the
slow scanner and the data acquisition, such that the first set of N
repeats can begin on any of the first N frames.

    \hypertarget{parameters-for-octocta-processing}{%
\subsection{Parameters for OCT/OCTA
processing}\label{parameters-for-octocta-processing}}

In addition to the raw data parameters, the code needs to know how to
process the OCT data. These parameters are of greater interest to OCT
scientists, and are subject to continual revision and refinement.

    \begin{Verbatim}[commandchars=\\\{\}]
{\color{incolor}In [{\color{incolor}4}]:} \PY{c+c1}{\PYZsh{} PROCESSING PARAMETERS}
        \PY{n}{mapping\PYZus{}coefficients} \PY{o}{=} \PY{p}{[}\PY{l+m+mf}{12.5e\PYZhy{}10}\PY{p}{,}\PY{o}{\PYZhy{}}\PY{l+m+mf}{12.5e\PYZhy{}7}\PY{p}{,}\PY{l+m+mf}{0.0}\PY{p}{,}\PY{l+m+mf}{0.0}\PY{p}{]}
        \PY{n}{dispersion\PYZus{}coefficients} \PY{o}{=} \PY{p}{[}\PY{l+m+mf}{0.0}\PY{p}{,}\PY{l+m+mf}{1.5e\PYZhy{}6}\PY{p}{,}\PY{l+m+mf}{0.0}\PY{p}{,}\PY{l+m+mf}{0.0}\PY{p}{]}
        
        \PY{n}{fft\PYZus{}oversampling\PYZus{}size} \PY{o}{=} \PY{l+m+mi}{4096}
        
        \PY{c+c1}{\PYZsh{} Cropping parameters:}
        \PY{n}{bscan\PYZus{}z1} \PY{o}{=} \PY{l+m+mi}{3147}
        \PY{n}{bscan\PYZus{}z2} \PY{o}{=} \PY{o}{\PYZhy{}}\PY{l+m+mi}{40} \PY{c+c1}{\PYZsh{} negative indices are relative to the end of the array}
        \PY{n}{bscan\PYZus{}x1} \PY{o}{=} \PY{l+m+mi}{0}
        \PY{n}{bscan\PYZus{}x2} \PY{o}{=} \PY{o}{\PYZhy{}}\PY{l+m+mi}{100} \PY{c+c1}{\PYZsh{} negative indices are relative to the end of the array}
\end{Verbatim}


    \hypertarget{pulling-and-processing-an-octa-frame}{%
\subsection{Pulling and processing an OCTA
frame}\label{pulling-and-processing-an-octa-frame}}

Let's say we want to process one OCTA frame, using the OCTRawData object
\texttt{src} defined above.

First, we need to get the raw spectra. Let's adopt the convention of
calling these \textbf{frames}. A frame has dimensions
\texttt{n\_k\ *\ n\_x}, where \texttt{n\_k} is the number of points in
the k-dimension (the vertical/first dimension, by convention) and
\texttt{n\_x} is the number of points in the fast scan dimension,
including repeats. Our B-scans are 500 pixels wide, and we have 5
repeats, so a single frame will contain 2500 A-scans. Remember that
Python, like every sane programming language, begins indices with 0, not
1. We'll get the first frame and see what it looks like.

    \begin{Verbatim}[commandchars=\\\{\}]
{\color{incolor}In [{\color{incolor}5}]:} \PY{n}{frame} \PY{o}{=} \PY{n}{src}\PY{o}{.}\PY{n}{get\PYZus{}frame}\PY{p}{(}\PY{l+m+mi}{0}\PY{p}{)}
        \PY{n}{plt}\PY{o}{.}\PY{n}{figure}\PY{p}{(}\PY{n}{dpi}\PY{o}{=}\PY{l+m+mi}{150}\PY{p}{)}
        \PY{n}{plt}\PY{o}{.}\PY{n}{imshow}\PY{p}{(}\PY{n}{frame}\PY{p}{,}\PY{n}{aspect}\PY{o}{=}\PY{l+s+s1}{\PYZsq{}}\PY{l+s+s1}{auto}\PY{l+s+s1}{\PYZsq{}}\PY{p}{)}
        \PY{n}{plt}\PY{o}{.}\PY{n}{colorbar}\PY{p}{(}\PY{p}{)}
        \PY{n}{plt}\PY{o}{.}\PY{n}{show}\PY{p}{(}\PY{p}{)}
\end{Verbatim}


    \begin{center}
    \adjustimage{max size={0.9\linewidth}{0.9\paperheight}}{output_12_0.png}
    \end{center}
    { \hspace*{\fill} \\}
    
    \hypertarget{oct-processing-pipeline}{%
\subsection{OCT processing pipeline}\label{oct-processing-pipeline}}

The next steps in the process are 1) DC-subtraction, 2) k-resampling, 3)
dispersion compensation, 4) windowing, and 5) FFTing (and oversampling)
the spectra into a B-scan. These are illustrated next.

    \begin{Verbatim}[commandchars=\\\{\}]
{\color{incolor}In [{\color{incolor}6}]:} \PY{n}{frame} \PY{o}{=} \PY{n}{blob}\PY{o}{.}\PY{n}{dc\PYZus{}subtract}\PY{p}{(}\PY{n}{frame}\PY{p}{)}
        \PY{n}{plt}\PY{o}{.}\PY{n}{figure}\PY{p}{(}\PY{n}{dpi}\PY{o}{=}\PY{l+m+mi}{150}\PY{p}{)}
        \PY{n}{plt}\PY{o}{.}\PY{n}{imshow}\PY{p}{(}\PY{n}{frame}\PY{p}{,}\PY{n}{aspect}\PY{o}{=}\PY{l+s+s1}{\PYZsq{}}\PY{l+s+s1}{auto}\PY{l+s+s1}{\PYZsq{}}\PY{p}{)}
        \PY{n}{plt}\PY{o}{.}\PY{n}{colorbar}\PY{p}{(}\PY{p}{)}
        \PY{n}{plt}\PY{o}{.}\PY{n}{show}\PY{p}{(}\PY{p}{)}
\end{Verbatim}


    \begin{center}
    \adjustimage{max size={0.9\linewidth}{0.9\paperheight}}{output_14_0.png}
    \end{center}
    { \hspace*{\fill} \\}
    
    \begin{Verbatim}[commandchars=\\\{\}]
{\color{incolor}In [{\color{incolor}7}]:} \PY{n}{frame} \PY{o}{=} \PY{n}{blob}\PY{o}{.}\PY{n}{k\PYZus{}resample}\PY{p}{(}\PY{n}{frame}\PY{p}{,}\PY{n}{mapping\PYZus{}coefficients}\PY{p}{)}
        \PY{n}{plt}\PY{o}{.}\PY{n}{figure}\PY{p}{(}\PY{n}{dpi}\PY{o}{=}\PY{l+m+mi}{150}\PY{p}{)}
        \PY{n}{plt}\PY{o}{.}\PY{n}{imshow}\PY{p}{(}\PY{n}{frame}\PY{p}{,}\PY{n}{aspect}\PY{o}{=}\PY{l+s+s1}{\PYZsq{}}\PY{l+s+s1}{auto}\PY{l+s+s1}{\PYZsq{}}\PY{p}{)}
        \PY{n}{plt}\PY{o}{.}\PY{n}{colorbar}\PY{p}{(}\PY{p}{)}
        \PY{n}{plt}\PY{o}{.}\PY{n}{show}\PY{p}{(}\PY{p}{)}
\end{Verbatim}


    \begin{center}
    \adjustimage{max size={0.9\linewidth}{0.9\paperheight}}{output_15_0.png}
    \end{center}
    { \hspace*{\fill} \\}
    
    \begin{Verbatim}[commandchars=\\\{\}]
{\color{incolor}In [{\color{incolor}8}]:} \PY{n}{frame} \PY{o}{=} \PY{n}{blob}\PY{o}{.}\PY{n}{dispersion\PYZus{}compensate}\PY{p}{(}\PY{n}{frame}\PY{p}{,}\PY{n}{dispersion\PYZus{}coefficients}\PY{p}{)}
        \PY{n}{plt}\PY{o}{.}\PY{n}{figure}\PY{p}{(}\PY{n}{dpi}\PY{o}{=}\PY{l+m+mi}{150}\PY{p}{)}
        \PY{n}{plt}\PY{o}{.}\PY{n}{imshow}\PY{p}{(}\PY{n}{np}\PY{o}{.}\PY{n}{abs}\PY{p}{(}\PY{n}{frame}\PY{p}{)}\PY{p}{,}\PY{n}{aspect}\PY{o}{=}\PY{l+s+s1}{\PYZsq{}}\PY{l+s+s1}{auto}\PY{l+s+s1}{\PYZsq{}}\PY{p}{)} \PY{c+c1}{\PYZsh{} need \PYZsq{}abs\PYZsq{} because dispersion compensation introduces imaginary component}
        \PY{n}{plt}\PY{o}{.}\PY{n}{colorbar}\PY{p}{(}\PY{p}{)}
        \PY{n}{plt}\PY{o}{.}\PY{n}{show}\PY{p}{(}\PY{p}{)}
\end{Verbatim}


    \begin{center}
    \adjustimage{max size={0.9\linewidth}{0.9\paperheight}}{output_16_0.png}
    \end{center}
    { \hspace*{\fill} \\}
    
    \begin{Verbatim}[commandchars=\\\{\}]
{\color{incolor}In [{\color{incolor}9}]:} \PY{n}{frame} \PY{o}{=} \PY{n}{blob}\PY{o}{.}\PY{n}{gaussian\PYZus{}window}\PY{p}{(}\PY{n}{frame}\PY{p}{,}\PY{l+m+mf}{0.9}\PY{p}{)}
        \PY{n}{plt}\PY{o}{.}\PY{n}{figure}\PY{p}{(}\PY{n}{dpi}\PY{o}{=}\PY{l+m+mi}{150}\PY{p}{)}
        \PY{n}{plt}\PY{o}{.}\PY{n}{imshow}\PY{p}{(}\PY{n}{np}\PY{o}{.}\PY{n}{abs}\PY{p}{(}\PY{n}{frame}\PY{p}{)}\PY{p}{,}\PY{n}{aspect}\PY{o}{=}\PY{l+s+s1}{\PYZsq{}}\PY{l+s+s1}{auto}\PY{l+s+s1}{\PYZsq{}}\PY{p}{)} \PY{c+c1}{\PYZsh{} need \PYZsq{}abs\PYZsq{} because dispersion compensation introduces imaginary component}
        \PY{n}{plt}\PY{o}{.}\PY{n}{colorbar}\PY{p}{(}\PY{p}{)}
        \PY{n}{plt}\PY{o}{.}\PY{n}{show}\PY{p}{(}\PY{p}{)}
\end{Verbatim}


    \begin{center}
    \adjustimage{max size={0.9\linewidth}{0.9\paperheight}}{output_17_0.png}
    \end{center}
    { \hspace*{\fill} \\}
    
    \hypertarget{lets-have-a-look-at-the-gaussian-window-just-for-fun-by-running-it-on-a-vector-of-ones}{%
\subsubsection{Let's have a look at the Gaussian window, just for fun,
by running it on a vector of
ones}\label{lets-have-a-look-at-the-gaussian-window-just-for-fun-by-running-it-on-a-vector-of-ones}}

    \begin{Verbatim}[commandchars=\\\{\}]
{\color{incolor}In [{\color{incolor}10}]:} \PY{n}{window\PYZus{}shape} \PY{o}{=} \PY{n}{blob}\PY{o}{.}\PY{n}{gaussian\PYZus{}window}\PY{p}{(}\PY{n}{np}\PY{o}{.}\PY{n}{ones}\PY{p}{(}\PY{n}{frame}\PY{o}{.}\PY{n}{shape}\PY{p}{[}\PY{l+m+mi}{0}\PY{p}{]}\PY{p}{)}\PY{p}{,}\PY{l+m+mf}{0.9}\PY{p}{)}
         \PY{n}{plt}\PY{o}{.}\PY{n}{figure}\PY{p}{(}\PY{n}{dpi}\PY{o}{=}\PY{l+m+mi}{150}\PY{p}{)}
         \PY{n}{plt}\PY{o}{.}\PY{n}{plot}\PY{p}{(}\PY{n}{window\PYZus{}shape}\PY{p}{)}
         \PY{n}{plt}\PY{o}{.}\PY{n}{show}\PY{p}{(}\PY{p}{)}
\end{Verbatim}


    \begin{center}
    \adjustimage{max size={0.9\linewidth}{0.9\paperheight}}{output_19_0.png}
    \end{center}
    { \hspace*{\fill} \\}
    
    \hypertarget{now-we-generate-a-b-scan-from-the-spectra}{%
\subsubsection{Now we generate a B-scan from the
spectra}\label{now-we-generate-a-b-scan-from-the-spectra}}

    \begin{Verbatim}[commandchars=\\\{\}]
{\color{incolor}In [{\color{incolor}11}]:} \PY{n}{bscan} \PY{o}{=} \PY{n}{blob}\PY{o}{.}\PY{n}{spectra\PYZus{}to\PYZus{}bscan}\PY{p}{(}\PY{n}{frame}\PY{p}{,}\PY{n}{oversampled\PYZus{}size}\PY{o}{=}\PY{n}{fft\PYZus{}oversampling\PYZus{}size}\PY{p}{,}\PY{n}{z1}\PY{o}{=}\PY{n}{bscan\PYZus{}z1}\PY{p}{,}\PY{n}{z2}\PY{o}{=}\PY{n}{bscan\PYZus{}z2}\PY{p}{)}
         \PY{n}{dB\PYZus{}bscan} \PY{o}{=} \PY{l+m+mi}{20}\PY{o}{*}\PY{n}{np}\PY{o}{.}\PY{n}{log10}\PY{p}{(}\PY{n}{np}\PY{o}{.}\PY{n}{abs}\PY{p}{(}\PY{n}{bscan}\PY{p}{)}\PY{p}{)}
         \PY{c+c1}{\PYZsh{} define rough contrast lims\PYZhy{}\PYZhy{}if our sensitivity is 90 dB and our dynamic range is 45 dB, then (45,90) will work.}
         \PY{n}{clim} \PY{o}{=} \PY{p}{(}\PY{l+m+mi}{45}\PY{p}{,}\PY{l+m+mi}{90}\PY{p}{)}
         \PY{n}{plt}\PY{o}{.}\PY{n}{figure}\PY{p}{(}\PY{n}{dpi}\PY{o}{=}\PY{l+m+mi}{150}\PY{p}{)}
         \PY{n}{plt}\PY{o}{.}\PY{n}{imshow}\PY{p}{(}\PY{n}{dB\PYZus{}bscan}\PY{p}{,}\PY{n}{clim}\PY{o}{=}\PY{p}{(}\PY{l+m+mi}{45}\PY{p}{,}\PY{l+m+mi}{90}\PY{p}{)}\PY{p}{,}\PY{n}{aspect}\PY{o}{=}\PY{l+s+s1}{\PYZsq{}}\PY{l+s+s1}{auto}\PY{l+s+s1}{\PYZsq{}}\PY{p}{)} \PY{c+c1}{\PYZsh{} need \PYZsq{}abs\PYZsq{} because dispersion compensation introduces imaginary component}
         \PY{n}{plt}\PY{o}{.}\PY{n}{colorbar}\PY{p}{(}\PY{p}{)}
         \PY{n}{plt}\PY{o}{.}\PY{n}{show}\PY{p}{(}\PY{p}{)}
\end{Verbatim}


    \begin{center}
    \adjustimage{max size={0.9\linewidth}{0.9\paperheight}}{output_21_0.png}
    \end{center}
    { \hspace*{\fill} \\}
    
    \hypertarget{now-we-have-to-reshape-the-compound-b-scan-into-a-stack-of-5-n_repeats-individual-b-scans}{%
\subsubsection{Now we have to reshape the compound B-scan into a stack
of 5 (n\_repeats) individual
B-scans}\label{now-we-have-to-reshape-the-compound-b-scan-into-a-stack-of-5-n_repeats-individual-b-scans}}

We'll check the shape of the stack (3D array), and then we'll visualize
the first one in the stack, as sanity checks.

    \begin{Verbatim}[commandchars=\\\{\}]
{\color{incolor}In [{\color{incolor}12}]:} \PY{n}{stack\PYZus{}complex} \PY{o}{=} \PY{n}{blob}\PY{o}{.}\PY{n}{reshape\PYZus{}repeats}\PY{p}{(}\PY{n}{bscan}\PY{p}{,}\PY{n}{n\PYZus{}repeats}\PY{p}{,}\PY{n}{x1}\PY{o}{=}\PY{n}{bscan\PYZus{}x1}\PY{p}{,}\PY{n}{x2}\PY{o}{=}\PY{n}{bscan\PYZus{}x2}\PY{p}{)}
         \PY{k}{print}\PY{p}{(}\PY{n}{stack\PYZus{}complex}\PY{o}{.}\PY{n}{shape}\PY{p}{)}
         \PY{c+c1}{\PYZsh{} remember that the original array bscan was complex; we used abs and log10 to visualize it before}
         \PY{n}{dB\PYZus{}first\PYZus{}bscan} \PY{o}{=} \PY{l+m+mi}{20}\PY{o}{*}\PY{n}{np}\PY{o}{.}\PY{n}{log10}\PY{p}{(}\PY{n}{np}\PY{o}{.}\PY{n}{abs}\PY{p}{(}\PY{n}{stack\PYZus{}complex}\PY{p}{[}\PY{p}{:}\PY{p}{,}\PY{p}{:}\PY{p}{,}\PY{l+m+mi}{0}\PY{p}{]}\PY{p}{)}\PY{p}{)}
         \PY{c+c1}{\PYZsh{} define rough contrast lims\PYZhy{}\PYZhy{}if our sensitivity is 90 dB and our dynamic range is 45 dB, then (45,90) will work.}
         \PY{n}{plt}\PY{o}{.}\PY{n}{figure}\PY{p}{(}\PY{n}{dpi}\PY{o}{=}\PY{l+m+mi}{150}\PY{p}{)}
         \PY{n}{plt}\PY{o}{.}\PY{n}{imshow}\PY{p}{(}\PY{n}{dB\PYZus{}first\PYZus{}bscan}\PY{p}{,}\PY{n}{clim}\PY{o}{=}\PY{p}{(}\PY{l+m+mi}{45}\PY{p}{,}\PY{l+m+mi}{90}\PY{p}{)}\PY{p}{,}\PY{n}{aspect}\PY{o}{=}\PY{l+s+s1}{\PYZsq{}}\PY{l+s+s1}{auto}\PY{l+s+s1}{\PYZsq{}}\PY{p}{)} \PY{c+c1}{\PYZsh{} need \PYZsq{}abs\PYZsq{} because dispersion compensation introduces imaginary component}
         \PY{n}{plt}\PY{o}{.}\PY{n}{colorbar}\PY{p}{(}\PY{p}{)}
         \PY{n}{plt}\PY{o}{.}\PY{n}{show}\PY{p}{(}\PY{p}{)}
\end{Verbatim}


    \begin{Verbatim}[commandchars=\\\{\}]
(909, 400, 5)

    \end{Verbatim}

    \begin{center}
    \adjustimage{max size={0.9\linewidth}{0.9\paperheight}}{output_23_1.png}
    \end{center}
    { \hspace*{\fill} \\}
    
    \hypertarget{lastly-well-convert-this-stack-of-complex-repeats-into-an-angiogram}{%
\subsubsection{Lastly, we'll convert this stack of complex repeats into
an
angiogram}\label{lastly-well-convert-this-stack-of-complex-repeats-into-an-angiogram}}

    \begin{Verbatim}[commandchars=\\\{\}]
{\color{incolor}In [{\color{incolor}13}]:} \PY{n}{phase\PYZus{}variance} \PY{o}{=} \PY{n}{blob}\PY{o}{.}\PY{n}{make\PYZus{}angiogram}\PY{p}{(}\PY{n}{stack\PYZus{}complex}\PY{p}{)}
         \PY{n}{plt}\PY{o}{.}\PY{n}{figure}\PY{p}{(}\PY{n}{dpi}\PY{o}{=}\PY{l+m+mi}{150}\PY{p}{)}
         \PY{n}{plt}\PY{o}{.}\PY{n}{imshow}\PY{p}{(}\PY{n}{phase\PYZus{}variance}\PY{p}{,}\PY{n}{clim}\PY{o}{=}\PY{p}{(}\PY{l+m+mi}{0}\PY{p}{,}\PY{l+m+mf}{0.2}\PY{o}{*}\PY{n}{np}\PY{o}{.}\PY{n}{pi}\PY{p}{)}\PY{p}{,}\PY{n}{aspect}\PY{o}{=}\PY{l+s+s1}{\PYZsq{}}\PY{l+s+s1}{auto}\PY{l+s+s1}{\PYZsq{}}\PY{p}{)} \PY{c+c1}{\PYZsh{} need \PYZsq{}abs\PYZsq{} because dispersion compensation introduces imaginary component}
         \PY{n}{plt}\PY{o}{.}\PY{n}{colorbar}\PY{p}{(}\PY{p}{)}
         \PY{n}{plt}\PY{o}{.}\PY{n}{show}\PY{p}{(}\PY{p}{)}
\end{Verbatim}


    \begin{center}
    \adjustimage{max size={0.9\linewidth}{0.9\paperheight}}{output_25_0.png}
    \end{center}
    { \hspace*{\fill} \\}
    
    \hypertarget{the-octa-processing-functions}{%
\subsubsection{The OCTA processing
functions}\label{the-octa-processing-functions}}

Obviously a lot of the work is buried in the OCTA processing functions,
and we'll eventually document all of those clearly as well. Here, for
example, is the dispersion compensation function:

    \begin{Verbatim}[commandchars=\\\{\}]
{\color{incolor}In [{\color{incolor}14}]:} \PY{k}{def} \PY{n+nf}{dispersion\PYZus{}compensate}\PY{p}{(}\PY{n}{spectra}\PY{p}{,}\PY{n}{coefficients}\PY{o}{=}\PY{p}{[}\PY{l+m+mf}{0.0}\PY{p}{,}\PY{l+m+mf}{1.5e\PYZhy{}6}\PY{p}{,}\PY{l+m+mf}{0.0}\PY{p}{,}\PY{l+m+mf}{0.0}\PY{p}{]}\PY{p}{)}\PY{p}{:}
             \PY{c+c1}{\PYZsh{} x\PYZus{}in specified on 1..N+1 to accord w/ Justin\PYZsq{}s code}
             \PY{c+c1}{\PYZsh{} fix this later, ideally as part of a greater effort}
             \PY{c+c1}{\PYZsh{} to define our meshes for mapping and dispersion compensation}
             \PY{c+c1}{\PYZsh{} on k instead of integer index}
             \PY{n}{x} \PY{o}{=} \PY{n}{np}\PY{o}{.}\PY{n}{arange}\PY{p}{(}\PY{l+m+mi}{1}\PY{p}{,}\PY{n}{spectra}\PY{o}{.}\PY{n}{shape}\PY{p}{[}\PY{l+m+mi}{0}\PY{p}{]}\PY{o}{+}\PY{l+m+mi}{1}\PY{p}{)}
             \PY{n}{dechirping\PYZus{}phasor} \PY{o}{=} \PY{n}{np}\PY{o}{.}\PY{n}{exp}\PY{p}{(}\PY{o}{\PYZhy{}}\PY{l+m+mi}{1j}\PY{o}{*}\PY{n}{np}\PY{o}{.}\PY{n}{polyval}\PY{p}{(}\PY{n}{coefficients}\PY{p}{,}\PY{n}{x}\PY{p}{)}\PY{p}{)}
             \PY{k}{return} \PY{p}{(}\PY{n}{spectra}\PY{o}{.}\PY{n}{T}\PY{o}{*}\PY{n}{dechirping\PYZus{}phasor}\PY{p}{)}\PY{o}{.}\PY{n}{T}
\end{Verbatim}



    % Add a bibliography block to the postdoc
    
    
    
    \end{document}
